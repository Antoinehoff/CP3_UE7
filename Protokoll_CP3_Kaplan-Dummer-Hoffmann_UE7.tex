\documentclass[%
	paper=A4,	% stellt auf A4-Papier
	pagesize,	% gibt Papiergröße weiter
	DIV=calc,	% errechnet Satzspiegel
	smallheadings,	% kleinere Überschriften
	ngerman		% neue Rechtschreibung
]{scrartcl}
\usepackage{BenMathTemplate}
\usepackage{BenTextTemplate}

\title{{\bf Wissenschaftliches Rechnen III / CP III}\\Übungsblatt 7}
\author{Tizia Kaplan (545978)\\Benjamin Dummer (532716)\\Antoine Hoffmann (426675)\\Gruppe 10}
\date{17.06.2016}

\begin{document}
\maketitle
Online-Version: \href{https://www.github.com/BeDummer/CP3_UE7}{\url{https://www.github.com/BeDummer/CP3_UE7}}

\section*{Aufgabe 7.1}


\section*{Aufgabe 7.2}


\section*{Anhänge}
\begin{itemize}
	\item Datei: \url{bla.cu} (Hauptprogramm)
\end{itemize}
\end{document}


%% Beispiel fuer Einbindung eines Bildes

%\begin{figure}
%  \centering
%  \includegraphics[width=.75\textwidth]{Dateiname}
%  \caption{Beschriftung}
%\end{figure}


%% Beispiel fuer Tabelle im Mathe-Modus

%\begin{eqnarray} \nonumber
%	\begin{array}{l|c|c}
% \mbox{Variablentyp} & \mbox{\tt int} & \mbox{\tt double}\\ \hline
% \mbox{Laufzeit [ms]} & 1.47 & 2.01  \\ 
% \mbox{Efficiency [\%]} & 100 & 100 \\
% \mbox{Throughput [GB/s]} & 47.8 & 69.4  \\
% \mbox{Occupancy} & 0.9985 & 0.9992
%	\end{array}
%\end{eqnarray}